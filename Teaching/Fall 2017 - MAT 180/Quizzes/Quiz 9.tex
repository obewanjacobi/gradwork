\documentclass[11pt,epsfig]{article}

\oddsidemargin=0in
\evensidemargin=0in
\textwidth=6.3in
\topmargin=-0.5in
\textheight=9in

\parindent=0in
\pagestyle{empty}

% C heck if we are compiling under latex or pdflatex, and include the
% appropriate graphics package
\usepackage{amsmath}
\ifx\pdftexversion\undefined
  \usepackage[dvips]{graphicx}
\else
  \usepackage[pdftex]{graphicx}
\fi

\input{testpoints}

\raggedbottom % Makes the bottom margin more flexible (helpful for pictures)

\begin{document}


%%%(change to appropriate class and semester)
Math 180-08;10;11 Fall

%%%(change to appropriate quiz type and date)
Quiz 9 October 25th \hspace{1.9in} {Name:} {\underline {\hspace{2.5in}}}
\vspace{2pc}

%%%(modify rules, time, points as appropriate)
Show all work to get full credit of this 10 point quiz. Please circle the section that you are in. All calculators are allowed, except for phones.
\vspace{2pc}

% problem
\begin{problem}{10}
Find the intervals where the function is increasing or decreasing, and local extrema of $f(x)$ where 
\begin{equation*}
f(x) = 2x^2 - 8x + 9
\end{equation*}
Sketching the graph is not necessary on this problem, but doing it may give you some leeway points. The same goes for the 2nd derivative test and concavity.
\vfill
\end{problem}






\showpoints
\end{document}
