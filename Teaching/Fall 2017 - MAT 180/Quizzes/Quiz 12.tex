\documentclass[11pt,epsfig]{article}

\oddsidemargin=0in
\evensidemargin=0in
\textwidth=6.3in
\topmargin=-0.5in
\textheight=9in

\parindent=0in
\pagestyle{empty}

% C heck if we are compiling under latex or pdflatex, and include the
% appropriate graphics package
\usepackage{amsmath}
\ifx\pdftexversion\undefined
  \usepackage[dvips]{graphicx}
\else
  \usepackage[pdftex]{graphicx}
\fi

\input{testpoints}

\raggedbottom % Makes the bottom margin more flexible (helpful for pictures)

\begin{document}


%%%(change to appropriate class and semester)
Math 180-08;10;11 Fall

%%%(change to appropriate quiz type and date)
Quiz 12 November 15th \hspace{1.9in} {Name:} {\underline {\hspace{2.5in}}}
\vspace{2pc}

%%%(modify rules, time, points as appropriate)
Show all work to get full credit of this 10 point quiz. Please circle the section that you are in. All calculators are allowed, except for phones. You may choose to do the 5 problems on the front or the single problem on the back in order to get the full 10 points. Please indicate which you would prefer to be graded by writing "front" or "back" on the top of the page. 
\vspace{2pc}

% problem
\begin{problem}{2}
Solve the following indefinite integral.
\begin{equation*}
\int 7dx
\end{equation*}
\vfill
\end{problem}

% problem
\begin{problem}{2}
Solve the following indefinite integral.
\begin{equation*}
\int 14xdx
\end{equation*}
\vfill
\end{problem}

% problem
\begin{problem}{2}
Solve the following indefinite integral.
\begin{equation*}
\int 15 x^2 dx
\end{equation*}
\vfill
\end{problem}

% problem
\begin{problem}{2}
Solve the following indefinite integral.
\begin{equation*}
\int 8 x^{1/3} dx
\end{equation*}
\vfill
\end{problem}

% problem
\begin{problem}{2}
Solve the following indefinite integral.
\begin{equation*}
\int 5 \mathrm{e}^x dx
\end{equation*}
\vfill
\end{problem}

\newpage

% problem
\begin{problem}{10}
Solve the following indefinite integral.
\begin{equation*}
\int x \mathrm{e} ^{x^2} dx
\end{equation*}
\vfill
\end{problem}






\showpoints
\end{document}
