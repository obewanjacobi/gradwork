% TEMPLATE FOR STANDARD QUIZ
\documentclass[11pt,epsfig]{article}

\oddsidemargin=0in
\evensidemargin=0in
\textwidth=6.3in
\topmargin=-0.5in
\textheight=9in

\parindent=0in
\pagestyle{empty}

% C heck if we are compiling under latex or pdflatex, and include the
% appropriate graphics package
\ifx\pdftexversion\undefined
  \usepackage[dvips]{graphicx}
\else
  \usepackage[pdftex]{graphicx}
\fi

\input{testpoints}

\usepackage{amsmath,amsfonts,amssymb,amstext,color,latexsym,graphics,tabulary,tabularx,graphicx,tikz}

\raggedbottom % Makes the bottom margin more flexible (helpful for pictures)

\begin{document}


%%%(change to appropriate class and semester)
Math 105-20 Summer Term II 

%%%(change to appropriate quiz type and date)
Equation sheet

%%%(modify rules, time, points as appropriate)

\vspace{2pc}

\begin{equation*}
FA=PMT\frac{((1+i)^m-1)}{i}
\end{equation*}
\vfill


\begin{equation*}
PV=PMT\frac{(1-(1+i)^{-m})}{i}
\end{equation*}
\vfill


\begin{equation*}
PMT=\frac{FA\times i}{((1+i)^m-1)}
\end{equation*}
\vfill


\begin{equation*}
PMT=\frac{PV\times i}{(1-(1+i)^{-m})}
\end{equation*}
\vfill



\begin{equation*}
m=log(1+\frac{FA \times i}{PMT}) \div log(1+i) 
\end{equation*}
\vfill


\begin{equation*}
m=-log(1-\frac{PV \times i}{PMT}) \div log(1+i) 
\end{equation*}
\vfill



\end{document}