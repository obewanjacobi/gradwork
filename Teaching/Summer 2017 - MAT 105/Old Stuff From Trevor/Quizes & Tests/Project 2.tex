% TEMPLATE FOR STANDARD QUIZ
\documentclass[11pt,epsfig]{article}

\oddsidemargin=0in
\evensidemargin=0in
\textwidth=6.3in
\topmargin=-0.5in
\textheight=9in

\parindent=0in
\pagestyle{empty}

% C heck if we are compiling under latex or pdflatex, and include the
% appropriate graphics package
\ifx\pdftexversion\undefined
  \usepackage[dvips]{graphicx}
\else
  \usepackage[pdftex]{graphicx}
\fi

\input{testpoints}

\usepackage{amsmath,amsfonts,amssymb,amstext,color,latexsym,graphics,tabulary,tabularx,graphicx,tikz}

\raggedbottom % Makes the bottom margin more flexible (helpful for pictures)

\begin{document}


%%%(change to appropriate class and semester)
Math 105-20 Summer Term II 

%%%(change to appropriate quiz type and date)
Project 2: DUE Thursday 7/02
\newline
Do all work on your own paper!!!!
\vspace{1pc}

%%%(modify rules, time, points as appropriate)

\vspace{0.5pc}






%problem
\begin{problem}{25}
A farmer owns a 160-acre farm, and plans to plant oats and/or soybeans on all or part of it. Seed for oats costs \$20 per acre, and seed for soybeans costs \$30 per acre. He can spend at most \$4440 for seed. Oats require 3 workdays per acre, and soybeans require 2 workdays per acre. There are a maximum of 396 workdays available. Each acre of oats produces 500 bushels and each acre of soybeans produces 200 bushels of grain, and he can store at most 60,000 bushels this year. In addition, because of crop rotation he cannot plant more than 140 acres of soybeans this year. If the farmer can make a profit of \$150 per acre on oats and \$200 per acre on soybeans, how many acres of each crop should he plant to maximize his profit?
\newline
A) Define the production variables that stand for the number of acres of each type crop to be planted.
\newline
B) Determine the profit equation for the total profit.
\newline 
C) Give a product/resource chart organizing the data about the products and the resources.
(Note: The farmer’s land is one of his resources; he cannot plant more than 160 total acres!
Also, he cannot plant more than 140 acres of soybeans!)
\newline
D) Give a mathematical formulation of the constraints in terms of your production variables.
\vfill
\end{problem}

%Problem
\begin{problem}{25}
\textbf{Carefully graph the feasible region from question \#1 on the attached graph paper. Be sure the scales on your axes start at 0 and the graph is large enough see the entire feasible region clearly.}
\newline
A)  Label \underline{all} the corner points \underline{and} determine their coordinates. \textbf{Show your work}.
\newline
B) Use the corner points to find the production policy (number of acres of each crop to be planted) that gives the maximum total profit and what that maximum profit is. Show your work and \textbf{describe your results in sentence form.}

\vfill
\end{problem}


%Problem
\begin{problem}{25}
The farmer is considering planting corn too. An acre of corn requires \$35 worth of seed, 2 workdays of labor, produces 450 bushels, and makes a profit of \$240 per acre. With this data, how much of each of the three crops should he plant to maximize his profit?
\newline
A) Setup the problem as you did in \#1, BUT DO NOT SOLVE.

\vfill
\end{problem}






\showpoints
\end{document}