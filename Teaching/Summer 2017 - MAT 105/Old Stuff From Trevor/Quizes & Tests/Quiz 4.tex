% TEMPLATE FOR STANDARD QUIZ
\documentclass[11pt,epsfig]{article}

\oddsidemargin=0in
\evensidemargin=0in
\textwidth=6.3in
\topmargin=-0.5in
\textheight=9in

\parindent=0in
\pagestyle{empty}

% C heck if we are compiling under latex or pdflatex, and include the
% appropriate graphics package
\ifx\pdftexversion\undefined
  \usepackage[dvips]{graphicx}
\else
  \usepackage[pdftex]{graphicx}
\fi

\input{testpoints}

\usepackage{amsmath,amsfonts,amssymb,amstext,color,latexsym,graphics,tabulary,tabularx,graphicx,tikz}

\raggedbottom % Makes the bottom margin more flexible (helpful for pictures)

\begin{document}


%%%(change to appropriate class and semester)
Math 105-21 Summer Term II 

%%%(change to appropriate quiz type and date)
Quiz 4: 6/15 \hspace{1.9in} {Name:} {\underline {\hspace{3.5in}}}
\vspace{1pc}
\begin{equation*}
PV=PMT\frac{1-(1+i)^{-m}}{i} \mbox{, } F=PMT\cdot i \mbox{ and } I=F-PV
\end{equation*}
\begin{equation*}
FA=PMT\frac{(1+i)^{m}-1}{i} \mbox{, } P=PMT\cdot i \mbox{ and } I=FA-P
\end{equation*}
\vspace{1pc}
%%%(modify rules, time, points as appropriate)
To receive full credit for this 25 point quiz, you must show ALL work.
\vspace{1pc}

% problem
\begin{problem}{5}
In your own words, describe the difference between the FA formula and the PV formula. Be sure to explicitly describe when you use each. 

\vfill
\end{problem}

% problem
\begin{problem}{5}
For the past ten years your uncle has been depositing \$1,000 at the end of each year in the savings account that pays 6.5\% compounded annually. What was the value of the account just after the tenth deposit? How much interest had been earned?

\vfill
\end{problem}

\newpage

% problem
\begin{problem}{5}
Best Buy offers a TV for \$300 down and \$15 per month for the next 24 months. If interest is charged at 9\% compounded monthly, find the cash value of the TV. How much total interest will the customer pay with this offer?

\vfill
\end{problem}

% problem
\begin{problem}{5}
A family wants to start a savings account for college expenses 18 years from now, and will make their first deposit at the end of this month. What equal monthly deposits in an account paying 5.0\% compounded monthly are needed in order to accumulate \$25,000 just after the last deposit 18 years from now? How much interest will the account have earned?  

\vfill
\end{problem}

% problem
\begin{problem}{5}
Suppose you purchace a new computer for \$1,500. You may finance the purchase by a loan for one year at an annual interest rate of 15\% compounded monthly. What would be your monthly payment if the loan is amortized? How much interest would you pay?

\vfill
\end{problem}

\showpoints
\end{document}