% TEMPLATE FOR STANDARD QUIZ
\documentclass[11pt,epsfig]{article}

\oddsidemargin=0in
\evensidemargin=0in
\textwidth=6.3in
\topmargin=-0.5in
\textheight=9in

\parindent=0in
\pagestyle{empty}

% C heck if we are compiling under latex or pdflatex, and include the
% appropriate graphics package
\ifx\pdftexversion\undefined
  \usepackage[dvips]{graphicx}
\else
  \usepackage[pdftex]{graphicx}
\fi

\input{testpoints}

\usepackage{amsmath,amsfonts,amssymb,amstext,color,latexsym,graphics,tabulary,tabularx,graphicx,tikz}

\raggedbottom % Makes the bottom margin more flexible (helpful for pictures)

\begin{document}


%%%(change to appropriate class and semester)
Math 105-20 Summer Term II 

%%%(change to appropriate quiz type and date)
Test 4: 7/07 \hspace{1.9in} {Name:} {\underline {\hspace{3.5in}}}
\vspace{2pc}

%%%(modify rules, time, points as appropriate)
To receive full credit for this 200 point Exam, you must show ALL work. Given the following preference schedule, answer the following questions.
\vspace{0.5pc}

   
% problem
\begin{problem}{25}
You are the member of a club with 37 members, and everybody in the club loves a good pizza. So you decided to choose a pizza place by having the members rank their favorite pizza places, with the following choices: Spinelli’s (S), Boombozz (B), Impellizzeri’s (I), and Wick’s (W). The following preference schedule results


 \begin{center}
 \begin{tabular}{ | l | c | c  |  c |  c | c |}
   \hline
   Number of Voters: & 13 & 10 & 8 & 5 & 1\\ \hline
   First & S & B & W & I & B \\ \hline
   Second & I & I & B & W & W \\ \hline
   Third & B & W & I & B & I \\ \hline
   Fourth & W & S & S & S & S \\ \hline
   \end{tabular}
  \end{center}
What is the winner using the method of Pairwise Comparisons? Does this voting method show a violation of Condorcet Fairness Criterion? Does this voting method show a violation of Majority Criterion? Explain your answer. 


\vfill
\end{problem}


% problem
\begin{problem}{25}
Using the voting preference from number 1. Find the winner of the election using the Borda Count Method. Does this voting method show a violation of Condorcet Fairness Criterion? Does this voting method show a violation of Majority Criterion? Explain your answer.
\vfill
\end{problem}

\newpage

 \begin{center}
 \begin{tabular}{ | l | c | c  |  c |  c | c |}
   \hline
   Number of Voters: & 13 & 10 & 8 & 5 & 1\\ \hline
   First & S & B & W & I & B \\ \hline
   Second & I & I & B & W & W \\ \hline
   Third & B & W & I & B & I \\ \hline
   Fourth & W & S & S & S & S \\ \hline
   \end{tabular}
  \end{center}
  

% problem
\begin{problem}{25}
Using the voting preference from number 1. Find the winner of the election using the Plurality. Does this voting method show a violation of Condorcet Fairness Criterion? Does this voting method show a violation of Majority Criterion? Explain your answer.

\vfill
\end{problem}

  
% problem
\begin{problem}{25}
Using the voting preference from number 1. Find the winner of the election using the Plurality with elimination. Does this voting method show a violation of Condorcet Fairness Criterion? Does this voting method show a violation of Majority Criterion? Explain your answer.

\vfill
\end{problem}

\newpage

%problem 
\begin{problem}{20}
A town has three districts, A, B, and C, and a force of 35 police officers. The population of the three districts are shown below. Apportion the police officers using the Hamilton Method
 \begin{center}
 \begin{tabular}{ | c | c | c | c | c | c |}
   \hline
   District: & Population & Standard Quota & Lower Quota & Extra Seat? & Hamilton Apport.\\ \hline
   A & 9,900 &  &  &  &  \\ \hline
   B & 6,615 &  &  &  &  \\ \hline
   C & 4,485 &  &  &  &  \\ \hline
    &  &  &  &  &  \\ \hline
   \end{tabular}
  \end{center}

\vfill
\end{problem}

%problem 
\begin{problem}{20}
A town has three districts, A, B, and C, and a force of 35 police officers. The population of the three districts are shown below. Apportion the police officers using the Hamilton Method
 \begin{center}
 \begin{tabular}{ | c | c | c | c | c | c |}
   \hline
   District: & Population & Standard Quota & Lower Quota & Extra Seat? & Hamilton Apport.\\ \hline
   A & 9,955 &  &  &  &  \\ \hline
   B & 6,915 &  &  &  &  \\ \hline
   C & 4,480 &  &  &  &  \\ \hline
    &  &  &  &  &  \\ \hline
   \end{tabular}
  \end{center}

\vfill
\end{problem}

%problem 
\begin{problem}{10}
What is the name of the paradox that the previous 2 examples demonstrates?

\vfill
\end{problem}

\newpage

%problem 
\begin{problem}{20}
A clinic has 225 nurses working four shifts. The number of nurses working each shift is to be apportioned using the Hamilton Method, according to the average number of patients in that shift. Apportion the nurses to the shifts using the Hamilton Method.

 \begin{center}
 \begin{tabular}{ | c | c | c | c | c | c |}
   \hline
   Shift: & Avg \# of Patients & Standard Quota & Lower Quota & Extra Seat? & Hamilton Apport.\\ \hline
   A & 869 &  &  &  &  \\ \hline
   B & 1025 &  &  &  &  \\ \hline
   C & 619 &  &  &  &  \\ \hline
   D & 187 &  &  &  &  \\ \hline
    &  &  &  &  &  \\ \hline
   \end{tabular}
  \end{center}

\vfill
\end{problem}

%problem 
\begin{problem}{20}
A clinic has 226 nurses working four shifts. The number of nurses working each shift is to be apportioned using the Hamilton Method, according to the average number of patients in that shift. Apportion the nurses to the shifts using the Hamilton Method.

 \begin{center}
 \begin{tabular}{ | c | c | c | c | c | c |}
   \hline
   Shift: & Avg \# of Patients & Standard Quota & Lower Quota & Extra Seat? & Hamilton Apport.\\ \hline
   A & 869 &  &  &  &  \\ \hline
   B & 1025 &  &  &  &  \\ \hline
   C & 619 &  &  &  &  \\ \hline
   D & 187 &  &  &  &  \\ \hline
    &  &  &  &  &  \\ \hline
   \end{tabular}
  \end{center}

\vfill
\end{problem}

%problem 
\begin{problem}{10}
What is the name of the paradox that the previous 2 examples demonstrates?

\vfill
\end{problem}

\showpoints
\end{document}