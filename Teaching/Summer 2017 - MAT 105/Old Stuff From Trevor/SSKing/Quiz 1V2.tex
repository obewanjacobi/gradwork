% TEMPLATE FOR STANDARD QUIZ



\documentclass[11pt,epsfig]{article}

\oddsidemargin=0in
\evensidemargin=0in
\textwidth=6.3in
\topmargin=-0.5in
\textheight=9in

\parindent=0in
\pagestyle{empty}

% C heck if we are compiling under latex or pdflatex, and include the
% appropriate graphics package
\ifx\pdftexversion\undefined
  \usepackage[dvips]{graphicx}
\else
  \usepackage[pdftex]{graphicx}
\fi

\input{testpoints}

\raggedbottom % Makes the bottom margin more flexible (helpful for pictures)

\begin{document}


%%%(change to appropriate class and semester)
Math 105-20 Summer Term II 

%%%(change to appropriate quiz type and date)
Quiz 2 \hspace{1.9in} {Name:} {\underline {\hspace{2.5in}}}
\vspace{2pc}

%%%(modify rules, time, points as appropriate)
Show all work to get full credit of this 25 point quiz.
\vspace{2pc}

% problem
\begin{problem}{5}
Solve for P:
\begin{equation*}
\$ 3000=P(1+0.02)\exp()
\end{equation*}
\vfill
\end{problem}

% problem
\begin{problem}{6}
Give the Equation of the Future value of Compound Interest (with more compoundings). Define all the variables you use. 
\vfill
\end{problem}

% problem
\begin{problem}{3}
The population of the world in the year 2000 was 6.0 billion. This was a 20\%
increase over the population in 1990. What was the population of the world in 1990?
\vfill
\end{problem}

\newpage

% problem
\begin{problem}{5}
If you deposit \$1000 at 6\% annual compount interest for 10 years, what will the value be if the compounding is:
\newline(a)Annual
\newline(b)Semi-Annual
\newline(c)Quarterly
\newline(d)Daily
\vfill
\end{problem}

\begin{problem}{5}
The advertisement shown below was found on a website that advertises current
interest rates for various certificates of deposit (CDs).
\newline
\centerline{\includegraphics[scale=0.6]{quiz1interest.jpg}}
\newline
Assuming that you invest the minimum deposit shown in the advertisement for this CD, what will be the value of this investment at the end of the five-year period based upon the APR?
Round your final answer to the nearest dollar.
\vfill
\end{problem}

\showpoints
\end{document}