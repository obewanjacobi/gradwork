\documentclass[12pt,leqno,psfig,openbib]{amsart}
\usepackage{amssymb,amsmath,graphicx,amsfonts,euscript}
\usepackage{fancyhdr}
\usepackage{epsfig}
\usepackage{setspace}
\usepackage{dsfont}
\usepackage[all]{xy}
\usepackage{amsmath}
\newcommand{\R}{\mathbb{R}}



\setlength{\textheight}{10.5in} \setlength{\textwidth}{6.6in}
\setlength{\oddsidemargin}{0.1in} \setlength{\evensidemargin}{0.1in}
\setlength{\parindent}{0.1in}
\setlength{\topmargin}{-0.5in} \setcounter{section}{0}
\setcounter{figure}{0} \setcounter{equation}{0}
\setlength{\leftmargin}{-0.5in} \setcounter{section}{0} 

\voffset=-0.05in

\numberwithin{equation}{section}



\begin{document}
\thispagestyle{empty}
\large\noindent Math 105-21, Summer 2016 \hfill Name: \underbar{\phantom{mmmmmmmmmmmmmmm}} \\{\bf  Exam 2}
\\Friday, 06/17/16 \hfill {\sl Score}: \underbar{\phantom{mmmmmmmmmm}} \\

\textbf{Instructions:} {\bf For full credit, show all your work and justify your answers. Unless specified otherwise, round your final answers to two decimal places. } \\\\
Formulas for future accumulation and present value: \hspace{2.5cm}  \\\\
$FA=PMT \frac{(1+i)^m-1}{i}$, $P=PMT\cdot m$, and $I=FA-P$ \\ \\
$PV=PMT \frac{1-(1+i)^{-m}}{i}$, $F=PMT \cdot m$, and $I=F-PV$\\
\vskip .250cm
 $m=\log \left( 1+\frac{FA \times i}{PMT}\right) \div \log (1+i)$
 \hspace{1.4cm}  $m=-\log \left( \frac{PV \times i}{PMT}-1 \right) \div \log (1+i)$  
\vskip .50cm
Payoff Am.= $PMT \frac{1-(1+i)^{-k}}{i}$.\\

\vskip .50cm

{\bf Problem 1} (15pts): Isabella wants to accumulate \$70,000 in 10 years by
making equal deposits at the end of each quarter in the account paying 7.2\%
interest compounded quarterly. What quarterly deposit should she make?
\vskip 10cm


(10 pts): How much of the \$70,000 she accumulates in the account will be interest she
has earned?

\newpage

{\bf Problem 2} (20 pts): You took a \$150,000 mortgage loan for 15 years at 3.6\% annual interest compounded monthly. Your monthly payment is \$919.50 You decided to terminate the loan after 10 years. What is the payoff amount? 
 \vskip 11cm


 {\bf Problem 3} (20 pts): Alice wants to accumulate \$70,000 by making equal deposits of \$300 each at the end of each month in the account paying 7.2\% interest compounded monthly. What is the required number of payments she needs in order to have {\bf{at least}} \$70,000 in the account? What will be the exact value of her account after the last required payment? 
 \vskip 8cm
 \newpage
 
 
{\bf Problem 4}  Brian wants to buy a \$240,000 house. He will pay  20\% down and finance the rest. 
\begin{enumerate}
\item[a).] (10 pts): How much will he have to finance?
\vskip 6cm
\item[b).] (20 pts): He is offered a loan for 30 years at 6.3\% compounded monthly, with no points. What will his monthly payment be?
\vskip 11cm
\item[c).] (15 pts): How much total interest will he pay if he makes all payments of the loan?
\end{enumerate}
\newpage


{\bf Problem 5} (30 pts): Denisa needs \$120,000 to complete the purchase of a house (she has already made the down payment). She has been offered a loan provided she pays $2.5$ points (percent) of the loan's value in fees. She is not going to pay cash for points.
 \begin{enumerate}
\item[a).]  How large a loan must she get so that it is just enough to cover both points and the \$120,000?
\vskip 6cm 
\item[b).]  If the interest rate is 4.1\%, what will her monlthy payments be on a 30 year mortgage?
\vskip 11cm
\end{enumerate} 

 {\bf Problem 6} (20 pts): Three months after his 20th birthday, Sebastian starts making quarterly payments of \$450 into a retirement account that pays 6.6\% compounded quarterly. He continues to do so for 45 more years until he is 65 years old. How much money will he then have in the account? 
\newpage

{\bf Problem 7} (40 pts): You buy a new computer for \$1,000. The store gives you a loan for 10 months at 12\% annual  interest compounded monthly. Your first nine monthly payments will be \$105.58 each. Find $i$ and complete the first two and last line of the amortization schedule:
\vskip .5cm
$i=\underline{\phantom{mmmm}}$.

\begin{center}
\renewcommand{\arraystretch}{1.5}
\begin{tabular}{|l|r|r|r|r|r|}
\hline
Payment \# & Beginning  & Amount of  & Amount of  & Principal & Balance after \\ 
            & Balance    & Payment &  Interest & Repaid & Payment  \\ \hline
1 & & & & & \\ \hline
2 & & & & & \\  \hline
. & & & & & \\ 
. & & & & & \\ 
. & & & & & \\ 
9 & & & & &104.53 \\ \hline
10  & & & & & \\ \hline
\end{tabular}
\end{center}
\vskip 9cm
What was the total amount of interest you paid on your loan?


\end{document}
