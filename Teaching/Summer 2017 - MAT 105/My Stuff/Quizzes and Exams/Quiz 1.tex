% TEMPLATE FOR STANDARD QUIZ



\documentclass[11pt,epsfig]{article}

\oddsidemargin=0in
\evensidemargin=0in
\textwidth=6.3in
\topmargin=-0.5in
\textheight=9in

\parindent=0in
\pagestyle{empty}

% C heck if we are compiling under latex or pdflatex, and include the
% appropriate graphics package
\ifx\pdftexversion\undefined
  \usepackage[dvips]{graphicx}
\else
  \usepackage[pdftex]{graphicx}
\fi

\input{testpoints}

\raggedbottom % Makes the bottom margin more flexible (helpful for pictures)

\begin{document}


%%%(change to appropriate class and semester)
Math 105-30 Summer Term III 

%%%(change to appropriate quiz type and date)
Quiz 1 July 10 \hspace{1.9in} {Name:} {\underline {\hspace{2.5in}}}
\vspace{2pc}

%%%(modify rules, time, points as appropriate)
Show all work to get full credit of this 25 point quiz. You are allowed to use your notes and book, but not your friends. All calculators are allowed, except for phones.
\vspace{2pc}

% problem
\begin{problem}{5}
Solve for P (round to the nearest hundreth):
\begin{equation}
\$ 4000=P(1+0.02)^{5}
\end{equation}
\vfill
\end{problem}

% problem
\begin{problem}{4}
Give the Equation of the Future value of Compound Interest (with more compoundings). Define all the variables you use. 
\vfill
\end{problem}

% problem
\begin{problem}{5}
The population of the world in the year 2000 was 6.0 billion. This was a 20\%
increase over the population in 1990. What was the population of the world in 1990?
\vfill
\end{problem}

\newpage

% problem
\begin{problem}{6}
If you deposit \$5000 at 8\% annual compount interest for 13 years, what will the value be if the compounding is:
\newline(a)Annual
\newline(b)Semi-Annual
\newline(c)Quarterly
\newline(d)Daily
\vfill
\end{problem}

\begin{problem}{5}
An ad for interest rates on certificate of deposit (CD) advertises a CD with a minimum deposit of \$10,000 and an APY of 1.15\%. 
If you invest the minimum amount, what will the value of this investment be at the end of a 5 year period? Round your answer to the nearest dollar. (Hint: We know $F=P(1+r)^t$, where does APY fit into this equation?)
\vfill
\end{problem}

\showpoints
\end{document}