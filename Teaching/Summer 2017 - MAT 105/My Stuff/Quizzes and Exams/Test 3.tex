% TEMPLATE FOR STANDARD QUIZ
\documentclass[11pt,epsfig]{article}

\oddsidemargin=0in
\evensidemargin=0in
\textwidth=6.3in
\topmargin=-0.5in
\textheight=9in

\parindent=0in
\pagestyle{empty}

% C heck if we are compiling under latex or pdflatex, and include the
% appropriate graphics package
\ifx\pdftexversion\undefined
  \usepackage[dvips]{graphicx}
\else
  \usepackage[pdftex]{graphicx}
\fi



%------------------------------------------------------------------
% PROBLEM, PART, AND POINT COUNTING...

% Create the problem number counter.  Initialize to zero.
\newcounter{problemnum}

% Specify that problems should be labeled with arabic numerals.
\renewcommand{\theproblemnum}{\arabic{problemnum}}


% Create the part-within-a-problem counter, "within" the problem counter.
% This counter resets to zero automatically every time the PROBLEMNUM counter
% is incremented.
\newcounter{partnum}[problemnum]

% Specify that parts should be labeled with lowercase letters.
\renewcommand{\thepartnum}{\alph{partnum}}

% Make a counter to keep track of total points assigned to problems...
\newcounter{totalpoints}

% Make counters to keep track of points for parts...
\newcounter{curprobpts}		% Points assigned for the problem as a whole.
\newcounter{totalparts}		% Total points assigned to the various parts.

% Make a counter to keep track of the number of points on each page...
\newcounter{pagepoints}
% This counter is reset each time a page is printed.

% This "program" keeps track of how many points appear on each page, so that
% the total can be printed on the page itself.  Points are added to the total
% for a page when the PART (not the problem) they are assigned to is specified.
% When a problem without parts appears, the PAGEPOINTS are incremented directly
% from the problem as a whole (CURPROBPTS).


%---------------------------------------------------------------------------


% The \problem environment first checks the information about the previous
% problem.  If no parts appeared (or if they were all assigned zero points,
% then it increments TOTALPOINTS directly from CURPROBPTS, the points assigned
% to the last problem as a whole.  If the last problem did contain parts, it
% checks to make sure that their point values total up to the correct sum.
% It then puts the problem number on the page, along with the points assigned
% to it.

\newenvironment{problem}[1]{
% STATEMENTS TO BE EXECUTED WHEN A NEW PROBLEM IS BEGUN:
%
% Increment the problem number counter, and set the current \ref value to that
% number.
\refstepcounter{problemnum}
%
% Add some vspace to separate from the last problem.
\vspace{0.15in} \par
%
\setcounter{curprobpts}{#1} \setcounter{totalparts}{0}	% Reset counters.
%
% Now put in the "announcement" on the page.
{\Large \bf \theproblemnum. \normalsize ({\it \arabic{curprobpts} point\null\ifnum \value{curprobpts} = 1\else s\fi}\/)}
}{
% STATEMENTS TO BE EXECUTED WHEN AN OLD PROBLEM IS ENDED:
%
% If no parts to problem, then increment TOTALPOINTS and PAGEPOINTS for the
% entire problem at once.
\ifnum \value{totalparts} = 0
	\addtocounter{totalpoints}{\value{curprobpts}}	% Add pts to total.
	\addtocounter{pagepoints}{\value{curprobpts}}	% Add pts to page total.
%
% If there were parts for the problem, then check to make sure they total up
% to the same number of points that the problem is worth. Issue a warning
% if not.
\else \ifnum \value{totalparts} = \value{curprobpts}
	\else \typeout{}
	\typeout{!!!!!!!   POINT ACCOUNTING ERROR   !!!!!!!!}
	\typeout{PROBLEM [\theproblemnum] WAS ALLOCATED \arabic{curprobpts} POINTS,}
	\typeout{BUT CONTAINS PARTS TOTALLING \arabic{totalparts} POINTS!}
	\typeout{}
	\fi
\fi
}


%---------------------------------------------------------------------------


% The \newpart command increments the part counter and displays an appropriate
% lowercase letter to mark the part.  It adds points to the point counter
% immediately.  If 0 points are specified, no point announcement is made.
% Otherwise, the announcement is in scriptsize italics.

\newcommand{\newpart}[1]
{
\refstepcounter{partnum}	% Set the current \ref value to the part number.
\hspace{0.25in}		% Indent the part by a quarter inch.
%
% If points are to be printed for this problem (signaled by point value > 0),
% then put them in in scriptsize italics.
\ifnum #1 > 0
	\makebox[0.5in][l]{{\bf \thepartnum.} {\bf ({\it #1 pt\ifnum #1 = 1\else s\fi\/}) \,\,}}
\else
	\makebox[0.25in][l]{({\bf \thepartnum})}
\fi
%
\hspace{0.1in}		% Lead the material away from the part "number".
%
\addtocounter{totalparts}{#1}	% Add points to totalparts for this problem.
\addtocounter{pagepoints}{#1}	% Add points to total for this page.
\addtocounter{totalpoints}{#1}	% Add points to total for entire test.
}


%---------------------------------------------------------------------------



% Just in case you want to skip some numbers in your test...

\newcommand{\skipproblem}[1]{\addtocounter{problemnum}{#1}}



%---------------------------------------------------------------------------


% The \showpoints command simply gives a count of the total points read in up to
% the location at which the command is placed.  Typically, one places one
% \showpoints command at the end of the latex file, just prior to the
% \end{document} command.  It can appear elsewhere, however.

\newcommand{\showpoints}
{
\typeout{}  
\typeout{====> A TOTAL OF \arabic{totalpoints} POINTS WERE READ.}
\typeout{}
}


%---------------------------------------------------------------------------



\usepackage{amsmath,amsfonts,amssymb,amstext,color,latexsym,graphics,tabulary,tabularx,graphicx,tikz}

\raggedbottom % Makes the bottom margin more flexible (helpful for pictures)

\begin{document}


%%%(change to appropriate class and semester)
Math 105-30 Summer Term III 

%%%(change to appropriate quiz type and date)
Test 3 \hspace{1.9in} {Name:} {\underline {\hspace{3.5in}}}
\vspace{1pc}

%%%(modify rules, time, points as appropriate)
To receive full credit for this 100 point test (maximum points possible being 125), you must show ALL work. And plainly list all variables you use.
\vspace{0.5pc}






%problem
\begin{problem}{10}
Given the following linear program:
\newline
\newline
Let S=the number of standard surfboards, C=number of competitive surfboards
\newline
\underline{Maximize:}\hspace*{0.5cm} $P=\$40S+\$75C$
\newline
\underline{Subject to:}\hspace*{0.45cm}
$6S+8C\leq 120$ 
\newline
\hspace*{2.4cm}$S+3C\leq 30$
\newline
\hspace*{2.4cm}$S \geq 0$, $C\geq 0$
\newline
\newline
A) Is it feasible to make 9 Standard surfboards and 8 competition surfboards? If so, what is the profit?
\vfill
\end{problem}





%problem
\begin{problem}{10}
Given the following linear program:
\newline
\newline
Let T = the number of Trick water skis, S = the number of competitive surfboards
\newline
\underline{Maximize:}\hspace*{0.5cm}  $P=\$40T+\$30S$
\newline
\underline{Subject to:}\hspace*{0.45cm} $6T+4S\leq 108$ 
\newline
\hspace*{2.4cm}$T+S\leq 24$
\newline
\hspace*{2.4cm}$T \geq 0$, $S\geq 0$
\newline
\newline
A) Is it feasible to make 5 trick water skis and 13 slalom water skis? If so, what is the profit?
\vfill
\end{problem}


\newpage

%problem
\begin{problem}{10}
The Florida Juice Company makes two types of fruit punch – Fruity and Tangy – by blending orange juice and apple juice into a mixture. The fruit punch is sold in 5-gallon bottles. A bottle of Fruity earns a profit of \$3, and a bottle of Tangy earns a \$2 profit. A bottle of Fruity requires 3 gallons of orange juice and 2 gallons of apple juice, while a bottle of Tangy requires 4 gallons of orange juice and 1 gallon of apple juice. There are 200 gallons of apple juice and 120 gallons of orange juice available.
\newline
A)What is the Linear Program for this question?
\vfill
\end{problem}


%problem
\begin{problem}{10}
Cardinal Candy makes a Rick Pitino mix and a Denny Crum mix. A box of Rick Mix takes 0.4 pounds of chocolate, 0.2 pounds of nuts, and 0.4 pounds of fruit, and sells for \$12.95. A box of Denny Mix takes 0.2 pounds of chocolate, 0.2 pounds of nuts, and 0.6 pounds of fruit, and sells for \$9.95. Chocolate costs \$6 per pound, nuts cost \$4 per pound, and fruit costs \$3 per pound. This week, Cardinal Candy has 44 pounds of chocolate, 26 pounds of nuts, and 72 pounds of fruit.
\newline
A)What is the Linear Program for this question?

\vfill
\end{problem}






\newpage

%problem
\begin{problem}{20}
A furniture manufacturer makes wooden tables and chairs. The production process involves two basic types of labor: carpentry and finishing. A table requires 2 hours of carpentry and 1 hour of finishing, whereas a chair requires 3 hours of carpentry and ½ hour of finishing. The profit is \$35 per table and \$20 per chair. The manufacturer's employees can supply a maximum of 108 hours of carpentry work and 20 hours of finishing work per day.
\newline
A)What is the Linear Program for this question?
\newline
B)Is it feasible to make 20 tables and no chairs? If so, what is the profit?
\vfill
\end{problem}






%problem
\begin{problem}{20}
A farmer has 70 acres of land available for planting soybeans or wheat. The cost of preparing the soil is \$60 per acre for soybeans and \$30 per acre for wheat. The number of workdays of labor required is 3 days per acre for soybeans and 4 days per acre for wheat. The farmer cannot spend more than \$1800 in preparation costs nor can he use more than 120 workdays. Each acre of soybeans yields a profit of \$180, while each acre of wheat yields a profit of \$100.
\newline
A)What is the Linear Program for this question?
\newline
B) Is it feasible to plant 0 acres soybeans and 0 acres of wheat?

\vfill
\end{problem}

\newpage

% problem
\begin{problem}{45}
A manufacturer of fiberglass camper tops for pickup trucks makes a compact model and a regular model. Each compact top requires 5 hours from the fabricating department and 2 hours from the finishing department. Each regular top requires 4 hours from the fabricating department and 3 hours from the finishing department. The maximum hours available per week in the fabricating department and finishing department, respectively, are 200 and 108. The company makes a profit of \$40 on each compact top and \$50 on each regular top.
\newline
\newline
A)What are the products? Assign production variables for each product.
\vfill
B)What is the profit function?
\vfill
C)Fill out a product Resources Chart, what is the constraints?
\vfill
D) Is it feasible to make 30 compact tops and 10 regular tops? If so, what is the profit?
\vfill 
\newpage
E) Graph the constraints and shade the feasible region.
\vfill
F) List the Corner Points.
\vfill
G) Solve/Maximize the linear program


\vfill
\end{problem}










\showpoints
\end{document}