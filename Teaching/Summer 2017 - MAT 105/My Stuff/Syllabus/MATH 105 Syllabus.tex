\documentclass[addpoints,12pt]{exam}
\usepackage{amsmath,amsfonts,amssymb,amstext,color,latexsym,graphics,tabulary,tabularx,graphicx,tikz}
\usepackage [english]{babel}

\begin{document}
\sffamily
\def\wl{\par \vspace{\baselineskip}}
\pagestyle{headandfoot}

\runningheadrule
\firstpageheader{ \sffamily MATH 105-30}{}{\sffamily Summer 2017 (Term III)}
\runningheader{}{}{}
\firstpagefooter{}{\sffamily Page \thepage}{}
\runningfooter{}{\sffamily Page \thepage}{}
\noindent \fbox{\begin{minipage}{0.98\textwidth}
\noindent {\centering \textbf{Contemporary Mathematics} \\
MTuWThF, 9:40 AM - 11:10 AM \\
Natural Sciences Building, Room 108 \\}
\begin{tabbing} 
\noindent  \textbf{Instructor:} \hspace{0.6cm} \= Jacob Townson \\
\textbf{E-mail:} \> jacob.townson@louisville.edu \\
\textbf{Office:} \> Natural Sciences Building, Room 213 \\
\textbf{Office Hours:} \> Available by appointment \& everyday after class for at least an hour\\
\end{tabbing}


\end{minipage}} 

\wl
\noindent
\textbf{Course Introduction} \\
Contemporary Mathematics is a 3 credit hour course. This course fulfills the mathematics content area of the General Education Program at the University of Louisville and fosters active learning by asking students to think critically and to communicate effectively. Students use mathematical modeling to solve practical problems. This is an applications based course in which we will study Interest, Periodic Payments, Linear Programming, Voting Theory, and Apportionment. These topics were chosen to show ways in which mathematics is applied to solve concrete problems in the modern world. 
\wl
\noindent
\textbf{Required Materials} 
\begin{itemize} \itemsep10pt \parskip-6pt \parsep0pt
\item \textit{Topics in Contemporary Mathematics}, 5th Edition (2011), by Wiley Williams
\item Scientific or graphing calculator, such as the TI-30X IIS (cell phones, tablets, computers, and any other device capable of communicating with others are not acceptable as calculators for tests and quizzes)
\end{itemize} 

\noindent \textbf{Online Course Management} \\
This course will make use of the Blackboard online course management system, as well as CardMail, the official University of Louisville e-mail provider, to provide announcements and supplementary materials for the course. Additionally, all grades will be posted on Blackboard. You should make yourself familiar with these resources immediately.
\wl
\noindent \textbf{Learning Outcomes} \\
Mathematics is concerned with solving real-world problems through mathematical methods.  Students will increase their critical thinking and communication skills. Through tests, quizzes, projects, and daily homework, students will demonstrate the following:
\begin{enumerate} \itemsep10pt \parskip-6pt \parsep0pt
\item Represent mathematical information symbolically, visually, and numerically.
\item Use arithmetic, algebraic, and geometric models to solve problems.
\item Interpret mathematical models, such as formulas, graphs, and tables.
\item Estimate and check answers to mathematical problems, determining reasonableness, alternatives, and correctness and completeness of solutions.
\end{enumerate}
\newpage



\noindent \textbf{Tests} \\
There will be four in-class tests, totaling 400 points (100 points each). Each test will be based upon the material in the corresponding chapter. They will be tentatively given on July 13, July 21, July 28, and August 8. There will not be a cumulative final examination. All tests will be open book and open notes. 
\wl
\noindent \textbf{Quizzes} \\
There will be 5 quizzes, and they will be worth 25 points each. I will only count the best 4 out of the 5 taken. All quiz dates will be announced during class at least one day in advance. Quizzes will be open note and open book.
\wl
\noindent \textbf{Projects} \\
There will be two projects assigned, and they will be worth 35 points each. These will tentatively be due on July 26 and August 2 respectively. Complete details for each project will be given approximately one week before the due date.
\wl
\noindent \textbf{Daily Homework} \\
Daily homework assignments will be given, but they will not be collected or graded. After each class, problems from the textbook (and possibly from other sources) that you should be capable of completing will be posted on Blackboard or emailed to you by 1 PM. You will be responsible for completing problems similar (or even identical) to your homework problems on tests and quizzes.
\wl
\noindent \textbf{Class Participation} \\
Attendance and participation will be required for this class for a total of 30 points in your final grade. 


\wl
\noindent \textbf{Grading} \\
In this course, you are guaranteed to have the opportunity to earn at least 600 points. The number of points needed to earn a particular grade, along with the number of points available for each graded component, is shown below.

\begin{table}[h] \sffamily \centering
\begin{tabular}{|c|c|lcc}
\cline{1-2} \cline{4-5} 
\textbf{Points Earned} & \textbf{Letter Grade} &             \multicolumn{1}{l|}{}            & \multicolumn{1}{c|}{\textbf{Component}}                    &    \multicolumn{1}{c|}{\textbf{Possible Points}}                                           \\ \cline{1-2} \cline{4-5} 
530 or more          & A                     & \multicolumn{1}{l|}{} & \multicolumn{1}{c|}{Tests} & \multicolumn{1}{c|}{400} \\ \cline{1-2} \cline{4-5} 
529 - 470              & B                     & \multicolumn{1}{l|}{} & \multicolumn{1}{c|}{Quizzes}              & \multicolumn{1}{c|}{100}                      \\ \cline{1-2} \cline{4-5} 
469 - 410                & C                     & \multicolumn{1}{l|}{} & \multicolumn{1}{c|}{Projects}            & \multicolumn{1}{c|}{70}                      \\ \cline{1-2} \cline{4-5} 
409 - 350                & D                     & \multicolumn{1}{l|}{} & \multicolumn{1}{c|}{Participation}            & \multicolumn{1}{c|}{30}                      \\ \cline{1-2} \cline{4-5}
349 or less          & F                     &         \hspace{1.5cm}                & \multicolumn{1}{l}{}                    &                                               \\ \cline{1-2}
\end{tabular}
\end{table}
\noindent Note that the grading guidelines established above are minimum guarantees. The instructor reserves the right to lower these cutoffs, including the use of plus/minus grades below the cutoffs, based upon the clustering of final grades and other factors.
\newpage
\noindent \textbf{Preparing for Tests and Quizzes} \\
Most problems on tests and quizzes will be derived from examples in the book, examples in the class, homework problems, and study guides that will be provided. In fact, there is a possibility that some of the problems will be exactly the same as the problems that you have already seen. Thus, using these resources is your best method to prepare for tests and quizzes.
\wl
\noindent \textbf{Extra Assistance} \\
Please feel free to ask questions during class and during office hours. If you cannot make it to my office hours, then feel free to make an appointment with me. Additionally, you can get free help from REACH, located on campus in Strickler Hall, Room 226 East.
\wl
\noindent \textbf{Policy for Missed Coursework} \\
Makeup and/or alternate assignments will be given for any student who is absent due to university-approved activities or a documented emergency. Any student anticipating an absence due to a university-approved activity should bring that to the attention of the instructor as soon as possible. In the event of an emergency, the instructor should be notified as soon as possible and be provided with appropriate documentation. The instructor reserves the right to determine what qualifies as a documented emergency. Any makeups will need to be done during my office hours; if my office hours do not fit your schedule, then you may be asked to complete it in the testing center, for a fee.
\wl
\noindent \textbf{Withdrawals, Incompletes, Audits, and Pass/Fail Grades} \\
The last day to withdraw from the course with a final grade of W is July 25. Do not expect to be able to withdraw from the course beyond this date. Grades of incomplete will only be available in extreme emergency situations and requires the approval of the department. This course may not be audited, and it may not be taken for pass/fail credit. 
\wl
\noindent \textbf{Academic Dishonesty} \\
While homework and projects may be done collaboratively (within certain guidelines for projects), tests and quizzes are to be individual efforts. Furthermore, only approved notes, books, and electronic devices (except for approved calculators) are to be used on tests and quizzes. Any evidence of academic dishonesty will result in disciplinary action, up to and including receiving a failing grade from the course.
\wl
\noindent \textbf{Statement for Students with Disabilities} \\
The University of Louisville is committed to providing access to programs and services for qualified students with disabilities. If you are a student with a disability and require accommodation to participate and complete requirements for this class, notify me immediately and contact the Disability Resource Center (Stevenson Hall, Room 119) for verification of eligibility and determination of specific accommodations.
\wl
\noindent \textbf{Classroom Etiquette} \\
This is an academic atmosphere. Arrive on time and stay for the entire class; if you know in advance that you must leave in the middle of class, inform the instructor prior to the start of class. Silence cell phones; if you are expecting an emergency call, inform the instructor prior to the start of class. Come prepared for class. No laptops, tablets, etc. No food or beverages without lids (beverages with lids are okay). If it is necessary for you to miss class, you should obtain notes, assignments, etc. from another student, and you should communicate clearly with the instructor before class concerning your documented excuse if a make-up will be necessary. Students are expected to adhere to the policies outlined in the Code of Student Conduct and the Code of Student Rights and Responsibilities found in the Student Handbook.
\wl
\noindent \textbf{Title IX/Clery Act Notification} \\
Sexual misconduct (including sexual harassment, sexual assault, and any other nonconsensual behavior of a sexual nature) and sex discrimination violate University policies. Students experiencing such behavior may obtain confidential support from the PEACC Program (852-2663), Counseling Center (852-6585), and Campus Health Services (852-6479). To report sexual misconduct or sex discrimination, contact the Dean of Students (852-5787) or University of Louisville Police (852-6111). Disclosure to University faculty or instructors of sexual misconduct, domestic violence, dating violence, or sex discrimination occurring on campus, in a University-sponsored program, or involving a campus visitor or University student or employee (whether current or former) is not confidential under Title IX. Faculty and instructors must forward such reports, including names and circumstances, to the University's Title IX officer.
For more information, see the Sexual Misconduct Resource Guide:
http://louisville.edu/hr/employeerelations/sexual-misconduct-brochure
\wl
\noindent \textbf{Disclaimer} \\
The instructor reserves the right to make changes in the syllabus when necessary to meet learning objectives, to compensate for missed classes, or for similar reasons.
\newpage

\noindent \textbf{Tentative Schedule} \\
We will follow this tentative schedule as best as we possibly can. It is very likely that we may get slightly ahead or behind schedule at times, and some content, including review sessions, may be removed from the schedule. Every effort will be made to keep the test and project due dates as scheduled. 
\begin{table}[bht] \sffamily \centering
    \begin{tabular}{|l|l|l|}
    \hline
    \textbf{Date}      & \textbf{Sections Covered}                                                           & \textbf{Tests/Projects} \\ \hline
    July 5    & \parbox[t]{5cm}{1.1 Percents \\ 1.2 Simple Interest   }                                      & ~               \\ \hline
    July 6   & \parbox[t]{8cm}{ 1.3 Annual Compound Interest\\1.4 Compounding More Often; APY}              & ~               \\ \hline
    July 7   & 1.5 Finding Present Value with Compound Interest                           & ~               \\ \hline
    July 10   & 1.6 Finding Interest Rate                                                  & ~               \\ \hline
    July 11   & 1.7 Length of Time for Investment Growth                                   & ~               \\ \hline
    July 12   &  \parbox[t]{9cm}{1.8 Consumer Price Index and Purchasing Power\\Chapter 1 Review}            & ~               \\ \hline
    July 13   & 2.1 Future Value of a Sequence of Payments                                 & Test 1          \\ \hline
    July 14   & 2.2 Present Value of a Sequence of Payments                                & ~               \\ \hline
    July 17   & 2.3 Finding the Required Periodic Payment                                & ~               \\ \hline
    July 18   & 2.4 Amortization Schedules                                                 & ~               \\ \hline
    July 19   &  2.6 Finding the Number of Payments                                & ~               \\ \hline
    July 20   & \parbox[t]{8cm}{2.5 Home Mortgages\\Chapter 2 Review}           &               \\ \hline
    July 21   &  3.1 Translating the Problem into Mathematics                       &    Test 2        \\ \hline
    July 24   & 3.1 Translating the Problem into Mathematics                               & ~               \\ \hline
    July 25   & 3.2 Solving a Linear Programming Problem Graphically                  &             \\ \hline
    July 26   & 3.2 Solving a Linear Programming Problem Graphically                       &Project 1 Due      \\ \hline
    July 27   & \parbox[t]{8cm}{3.3 The Simplex Method\\Chapter 3 Review}           & ~               \\ \hline
    July 28  & 4.1 First Examples of Voting Methods &  Test 3              \\ \hline   
    July 31  & 4.2 More Involved Voting Methods         &                                      \\ \hline
    August 1  &  4.3 Is There an Ideal Voting Method?     &                 \\ \hline
    August 2  & 4.4 The Hamilton Method                                                    & Project 2 Due               \\ \hline
    August 3  & 4.5 Early Divisor Methods                                                  & ~               \\ \hline
    August 4  & 4.7 The Search for an Ideal Apportionment Method                           &               \\ \hline
    August 7 &  \parbox[t]{8cm}{4.6 Measuring Unfairness in Apportionments\\Chapter 4 Review}              & ~               \\ \hline
    August 8 & ~                                                                          & Test 4          \\ \hline
    \end{tabular}
\end{table}

\end{document}