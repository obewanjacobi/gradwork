\documentclass[addpoints,12pt]{exam}
\usepackage{amsmath,amsfonts,amssymb,amstext,color,latexsym,graphics,tabulary,tabularx,graphicx,tikz}
\usepackage [english]{babel}

\begin{document}
\sffamily
\def\wl{\par \vspace{\baselineskip}}
\pagestyle{headandfoot}

\runningheadrule
\firstpageheader{ \sffamily MATH 105-30}{}{\sffamily Summer 2017 (Term III)}
\runningheader{}{}{}
\firstpagefooter{}{\sffamily Page \thepage}{}
\runningfooter{}{\sffamily Page \thepage}{}
\noindent \fbox{\begin{minipage}{0.98\textwidth}
\noindent {\centering \textbf{Contemporary Mathematics} \\
MTuWThF, 9:40 AM - 11:10 AM \\
Natural Sciences Building, Room 108 \\}
\begin{tabbing} 
\noindent  \textbf{Instructor:} \hspace{0.6cm} \= Jacob Townson \\
\textbf{E-mail:} \> jacob.townson@louisville.edu \\
\textbf{Office:} \> Natural Sciences Building, Room 213 \\
\textbf{Office Hours:} \> Available by appointment \& the following:\\
\textbf{} \> \underline{June 2nd - June 18th:} \hspace{0.8cm}	\underline{June 22- July 7:} \\ 
\textbf{} \> Wed/Fri: 11:30-12:30   \hspace{0.9cm}  Mon/Wed/Fri: 11:30-12:30 \\
\end{tabbing}


\end{minipage}} 

\wl
\noindent
\textbf{Course Introduction} \\
Contemporary Mathematics is a 3 credit hour course. This course fulfills the mathematics content area of the General Education Program at the University of Louisville and fosters active learning by asking students to think critically and to communicate effectively. Students use mathematical modeling to solve practical problems. This is an applications based course in which we will study Interest, Periodic Payments, Linear Programming, Voting Theory, and Apportionment. These topics were chosen to show ways in which mathematics is applied to solve concrete problems in the modern world. 
\wl
\noindent
\textbf{Required Materials} 
\begin{itemize} \itemsep10pt \parskip-6pt \parsep0pt
\item \textit{Topics in Contemporary Mathematics}, 5th Edition (2011), by Wiley Williams
\item Scientific or graphing calculator, such as the TI-30X IIS (cell phones, tablets, computers, and any other device capable of communicating with others are not acceptable as calculators for tests and quizzes)
\end{itemize} 

\noindent \textbf{Online Course Management} \\
This course will make use of the Blackboard online course management system, as well as CardMail, the official University of Louisville e-mail provider, to provide announcements and supplementary materials for the course. Additionally, all grades will be posted on Blackboard. You should make yourself familiar with these resources immediately.
\wl
\noindent \textbf{Learning Outcomes} \\
Mathematics is concerned with solving real-world problems through mathematical methods.  Students will increase their critical thinking and communication skills. Through tests, quizzes, projects, and daily homework, students will demonstrate the following:
\begin{enumerate} \itemsep10pt \parskip-6pt \parsep0pt
\item Represent mathematical information symbolically, visually, and numerically.
\item Use arithmetic, algebraic, and geometric models to solve problems.
\item Interpret mathematical models, such as formulas, graphs, and tables.
\item Estimate and check answers to mathematical problems, determining reasonableness, alternatives, and correctness and completeness of solutions.
\end{enumerate}
\newpage



\noindent \textbf{Tests} \\
There will be four in-class tests, totaling 650 points. Each test will be based upon the material in the corresponding chapter. They will be tentatively given on June 10, June 18, June 25, and July 7. There will not be a cumulative final examination.
\wl
\noindent \textbf{Quizzes} \\
There will eight quizzes, and they will be worth 25 points each. Most will be in-class quizzes, but some may be take-home. All in-class quiz dates will be announced during class at least one day in advance. In-class quizzes will be open note and open book. Take-home quizzes will be due the day after they are assigned, unless indicated otherwise.
\wl
\noindent \textbf{Projects} \\
There will be two projects assigned, and they will be worth 75 points each. The first will be due on June 23 and the second will be due on June 30. Complete details for each project will be given approximately one week before the due date.
\wl
\noindent \textbf{Daily Homework} \\
Daily homework assignments will be given, but they will not be collected or graded. After each class, problems from the textbook (and possibly from other sources) that you should be capable of completing will be posted on Blackboard by noon. You will be responsible for completing problems similar (or even identical) to your homework problems on tests and quizzes.


\wl
\noindent \textbf{Grading} \\
In this course, you are guaranteed to have the opportunity to earn at least 1000 points. The number of points needed to earn a particular grade, along with the number of points available for each graded component, is shown below.

\begin{table}[h] \sffamily \centering
\begin{tabular}{|c|c|lcc}
\cline{1-2} \cline{4-5} 
\textbf{Points Earned} & \textbf{Letter Grade} &             \multicolumn{1}{l|}{}            & \multicolumn{1}{c|}{\textbf{Component}}                    &    \multicolumn{1}{c|}{\textbf{Possible Points}}                                           \\ \cline{1-2} \cline{4-5} 
900 or more          & A                     & \multicolumn{1}{l|}{} & \multicolumn{1}{c|}{Tests} & \multicolumn{1}{c|}{650} \\ \cline{1-2} \cline{4-5} 
800 - 899              & B                     & \multicolumn{1}{l|}{} & \multicolumn{1}{c|}{Quizzes}              & \multicolumn{1}{c|}{200}                      \\ \cline{1-2} \cline{4-5} 
700 - 799                & C                     & \multicolumn{1}{l|}{} & \multicolumn{1}{c|}{Projects}            & \multicolumn{1}{c|}{150}                      \\ \cline{1-2} \cline{4-5} 
600 - 699                & D                     & \multicolumn{1}{l}{} & \multicolumn{1}{c}{}            & \multicolumn{1}{c}{}                      \\ \cline{1-2}
599 or less          & F                     &         \hspace{1.5cm}                & \multicolumn{1}{l}{}                    &                                               \\ \cline{1-2}
\end{tabular}
\end{table}
\noindent Note that the grading guidelines established above are minimum guarantees. The instructor reserves the right to lower these cutoffs, including the use of plus/minus grades below the cutoffs, based upon the clustering of final grades and other factors.
\newpage
\noindent \textbf{Preparing for Tests and Quizzes} \\
Most problems on tests and quizzes will be derived from examples in the book, examples in the class, homework problems, and study guides that will be provided. In fact, there is a possibility that some of the problems will be exactly the same as the problems that you have already seen. Thus, using these resources is your best method to prepare for tests and quizzes.
\wl
\noindent \textbf{Extra Assistance} \\
Please feel free to ask questions during class and during office hours. If you cannot make it to my office hours, then feel free to make an appointment with me. Additionally, you can get free help from REACH, located on campus in Strickler Hall, Room 226 East.
\wl
\noindent \textbf{Policy for Missed Coursework} \\
Makeup and/or alternate assignments will be given for any student who is absent due to university-approved activities or a documented emergency. Any student anticipating an absence due to a university-approved activity should bring that to the attention of the instructor as soon as possible. In the event of an emergency, the instructor should be notified as soon as possible and be provided with appropriate documentation. The instructor reserves the right to determine what qualifies as a documented emergency. Any makeups will need to be done during my office hours; if my office hours do not fit your schedule, then you may be asked to complete it in the testing center, for a fee.
\wl
\noindent \textbf{Withdrawals, Incompletes, Audits, and Pass/Fail Grades} \\
The last day to withdraw from the course with a final grade of W is June 23. Do not expect to be able to withdraw from the course beyond this date. Grades of incomplete will only be available in extreme emergency situations and requires the approval of the department. This course may not be audited, and it may not be taken for pass/fail credit. 
\wl
\noindent \textbf{Academic Dishonesty} \\
While homework and projects may be done collaboratively (within certain guidelines for projects), tests and quizzes are to be individual efforts. Furthermore, unless other directives are otherwise given, no notes, books, or electronic devices (except for approved calculators) are to be used on tests and quizzes. Any evidence of academic dishonesty will result in disciplinary action, up to and including receiving a failing grade from the course.
\wl
\noindent \textbf{Statement for Students with Disabilities} \\
The University of Louisville is committed to providing access to programs and services for qualified students with disabilities. If you are a student with a disability and require accommodation to participate and complete requirements for this class, notify me immediately and contact the Disability Resource Center (Stevenson Hall, Room 119) for verification of eligibility and determination of specific accommodations.
\wl
\noindent \textbf{Disclaimer} \\
The instructor reserves the right to make changes in the syllabus when necessary to meet learning objectives, to compensate for missed classes, or for similar reasons.
\newpage

\noindent \textbf{Tentative Schedule} \\
We will follow this tentative schedule as best as we possibly can. It is very likely that we may get slightly ahead or behind schedule at times, and some content, including review sessions, may be removed from the schedule. Every effort will be made to keep the test and project due dates as scheduled. 
\begin{table}[bht] \sffamily \centering
    \begin{tabular}{|l|l|l|}
    \hline
    \textbf{Date}      & \textbf{Sections Covered}                                                           & \textbf{Tests/Projects} \\ \hline
    June 2    & \parbox[t]{5cm}{1.1 Percents \\ 1.2 Simple Interest   }                                      & ~               \\ \hline
    June 3   & \parbox[t]{8cm}{ 1.3 Annual Compound Interest\\1.4 Compounding More Often; APY}              & ~               \\ \hline
    June 4   & 1.5 Finding Present Value with Compound Interest                           & ~               \\ \hline
    June 5   & 1.6 Finding Interest Rate                                                  & ~               \\ \hline
    June 8   & 1.7 Length of Time for Investment Growth                                   & ~               \\ \hline
    June 9   &  \parbox[t]{9cm}{1.8 Consumer Price Index and Purchasing Power\\Chapter 1 Review}            & ~               \\ \hline
    June 10   & 2.1 Future Value of a Sequence of Payments                                 & Test 1          \\ \hline
    June 11   & 2.2 Present Value of a Sequence of Payments                                & ~               \\ \hline
    June 12   & 2.3 Finding the Required Periodic Payment                                & ~               \\ \hline
    June 15   & 2.4 Amortization Schedules                                                 & ~               \\ \hline
    June 16   &  2.6 Finding the Number of Payments                                & ~               \\ \hline
    June 17   & \parbox[t]{8cm}{2.5 Home Mortgages\\Chapter 2 Review}           &               \\ \hline
    June 18   &  3.1 Translating the Problem into Mathematics                       &    Test 2        \\ \hline
    June 19   & 3.1 Translating the Problem into Mathematics                               & ~               \\ \hline
    June 22   & 3.2 Solving a Linear Programming Problem Graphically                  &             \\ \hline
    June 23   & 3.2 Solving a Linear Programming Problem Graphically                       &Project 1 Due      \\ \hline
    June 24   & \parbox[t]{8cm}{3.3 The Simplex Method\\Chapter 3 Review}           & ~               \\ \hline
    June 25  & 4.1 First Examples of Voting Methods &  Test 3              \\ \hline   
    June 26  & 4.2 More Involved Voting Methods         &                                      \\ \hline
    June 29  &  4.3 Is There an Ideal Voting Method?     &                 \\ \hline
    June 30  & 4.4 The Hamilton Method                                                    & Project 2 Due               \\ \hline
    July 1  & 4.5 Early Divisor Methods                                                  & ~               \\ \hline
    July 2  & 4.7 The Search for an Ideal Apportionment Method                           &               \\ \hline
    July 6 &  \parbox[t]{8cm}{4.6 Measuring Unfairness in Apportionments\\Chapter 4 Review}              & ~               \\ \hline
    July 7 & ~                                                                          & Test 4          \\ \hline
    \end{tabular}
\end{table}
\end{document}