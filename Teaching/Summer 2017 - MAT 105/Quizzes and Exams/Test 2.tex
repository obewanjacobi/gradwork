% TEMPLATE FOR STANDARD QUIZ
\documentclass[11pt,epsfig]{article}

\oddsidemargin=0in
\evensidemargin=0in
\textwidth=6.3in
\topmargin=-0.5in
\textheight=9in

\parindent=0in
\pagestyle{empty}

% C heck if we are compiling under latex or pdflatex, and include the
% appropriate graphics package
\ifx\pdftexversion\undefined
  \usepackage[dvips]{graphicx}
\else
  \usepackage[pdftex]{graphicx}
\fi

\input{testpoints}

\usepackage{amsmath,amsfonts,amssymb,amstext,color,latexsym,graphics,tabulary,tabularx,graphicx,tikz}

\raggedbottom % Makes the bottom margin more flexible (helpful for pictures)

\begin{document}


%%%(change to appropriate class and semester)
Math 105-30 Summer Term III 

%%%(change to appropriate quiz type and date)
Test 2: 7/21 \hspace{1.9in} {Name:} {\underline {\hspace{3.5in}}}
\vspace{2pc}

%%%(modify rules, time, points as appropriate)
To receive full credit for this 100 point test (maximum points possible being 120), you must show ALL work. And plainly list all variables you use. For all answers, round to the nearest hundreth, or in some cases hundreth of a percent. If finding $m$ values, round $m$ to the nearest hundreth to convert to years and months. Feel free to use the backs of pages
\vspace{2pc}



% problem
\begin{problem}{12}
Suppose you want to buy a computer for \$5,000 to play Skyrim on. But you took an arrow to the knee and can't afford it out of pocket due to your medical expenses. Luckily Dell lets you make 24 equal monthly payments at 9\% annual interest compounded monthly. What is your monthly payment? How much interest will you pay?

\vfill
\end{problem}

% problem
\begin{problem}{12}
In your own words, describe the difference between compound interest problems in Chapter 1 material, and compound interest problems in chapter 2 material .

\vfill
\end{problem}

\newpage

% problem
\begin{problem}{12}
For the past ten years your uncle has been depositing \$500 at the end of each year in the savings account that pays 3.5\% compounded annually. What was the value of the account just after the tenth deposit? How much interest had been earned? (This is an amazing savings account!)

\vfill
\end{problem}

% problem
\begin{problem}{12}
Best Buy offers a TV for \$200 down and \$25 per month for the next 12 months. If interest is charged at 9\% compounded monthly, find the cash value of the TV. How much total interest will the customer pay with this offer?

\vfill
\end{problem}

\newpage

% problem
\begin{problem}{12}
A family wants to start a savings account for college expenses 18 years from now, and will make their first deposit at the end of this month. What equal monthly deposits in an account paying 3.6\% compounded monthly are needed in order to accumulate \$25,000 just after the last deposit 18 years from now? How much interest will the account have earned?  

\vfill
\end{problem}

% problem
\begin{problem}{12}
A car dealer offers to sell you a new car if you trade in your old car worth \$4,500, make a down payment of \$1,600, and make monthly payments of \$200 for four years. Assuming you can borrow money at 7.5\% compounded monthly, what is the cash value of the deal he is offering on the new car? How much interest would you pay?
  
\vfill
\end{problem}


\newpage


% problem
\begin{problem}{12}
Suppose you purchase a new tablet for \$960. You may finance the purchase by a loan for one year at an annual interest rate of 15\% compounded monthly. What would be your monthly payment if the loan is amortized? How much interest would you pay?

\vfill
\end{problem}

% problem
\begin{problem}{12}
John takes out a 30 year mortgage for \$80,000 at 7.2\% compounded monthly. His monthly payment is \$543.03. He decides to pay an additional \$100 each month until the loan is repaid. How long (in years and months) will it take for him to repay the loan in this fashion? (you can round your $m$ value to the nearest hundreth and find your answers from there.)


\vfill
\end{problem}

\newpage

% problem
\begin{problem}{12}
Frank plans to deposit \$350 per month into his retirement account that pays 5.5\% annual interest compounded monthly. If he does so, and the account continues to pay interest at the current rate, how long will it be until his account has a value of \$1,000,000? About how much total interest will he have earned?

\vfill
\end{problem}

% problem
\begin{problem}{12}
Suppose you decide to purchase a home with a selling price of \$115,000 by paying 20\% down and financing the remaining balance with monthly payments over 30 years at 7.2\% interest, compounded monthly
\newline
A) What is the total amount you will finance in the loan?
\newline
B) What will be your monthly payment? How much interest is your first payment?
\newline
C) What is the total amount of interest you will pay if you make the 30 years of payments?


\vfill
\end{problem}



\showpoints
\end{document}